\chapter{Conclusiones}
\label{cap:capitulo5}

\begin{flushright}
\begin{minipage}[]{10cm}
\emph{El talento es importante, pero la perseverancia es más importante.}\\
\end{minipage}\\

Angela Duckworth, \textit{Grit}\\
\end{flushright}

\vspace{1cm}

En este último capítulo se presentan los objetivos alcanzados y las conclusiones que se han obtenido, junto con las competencias, habilidades y conocimientos adquiridos durante el desarrollo de este proyecto, incluyendo además posibles líneas de trabajo futuras.


\section{Objetivos cumplidos}
\label{sec:objetivos_cumplidos}


Con el desarrollo de este proyecto se han logrado todos los objetivos establecidos en el Capítulo \ref{cap:capitulo3}. El objetivo principal era diseñar un robot móvil de bajo coste con piezas 3D que sea capaz de moverse en interiores hacia un destino final mediante comandos de voz y que fuera capaz de localizarse mediante dispositivos Wi-Fi en un mapa establecido previamente. A su vez, el robot debía orientarse adecuadamente, navegar correctamente por el mapa detectando obstáculos y que el sistema fuera portable a otros entornos de la manera más sencilla posible y que todas estas características funcionasen en un CPU de baja capacidad de cómputo. Todo ello se ha conseguido de manera exitosa.\\


También se han alcanzado los objetivos específicos que se propusieron. Uno de ellos era entrenar una red neuronal de audios por voz para enseñar a la red a clasificar e interpretar las órdenes dadas por el usuario. Otro era conseguir que el robot se localizara en el mapa a partir de 4 dispositivos Wi-Fi.\\

En cuanto al resto de objetivos secundarios, se consiguió que el robot se orientase correctamente y que detectase obstáculos mediante el sensor MPU9250 y HC-SR04 respectivamente. Para conseguir que fuera portable a otros entornos, se usó la herramienta \hyperlink{GIMP}{GIMP} con la cual se puede diseñar el mapa binario de una manera fácil y sencilla y que el robot navegue por el mapa a parir de la matriz del mismo y del algoritmo de navegación A* para encontrar la ruta más corta de manera eficiente hacia el destino final. Por último, se consiguió combinar todos los datos de navegación y localización mediante la librería threading a partir del control y la ejecución de diferentes hilos ejecutados en paralelo. Al usar una placa Raspberry Pi 4, se ha conseguido que el sistema implementado sea accesible económicamente para cualquier usuario.\\


La lectura de los sensores se han ido volcando en una doble cola de tamaño 1 almacenando así el último valor y eliminando automáticamente los anteriores ya que si no se especifica el tamaño máximo de cola, puede llegar a consumir mucha memoria porque se generan muchos datos en poco tiempo.\\


Para saber cuántos dispositivos Wi-Fi usar, se probó el sistema en cuatro escenarios de prueba donde se evaluaron las hipótesis de si la precisión de las IPS Wi-Fi HaLow escalaba linealmente con el número de balizas y la precisión aumentaba cuando la distancia mínima entre dos posiciones del grid aumentaba. Ambas hipótesis se confirmaron a partir de obtener numerosas muestras en cada punto del grid y de aplicar modelos de aprendizaje automáticos y comprobar cómo clasificaban los datos en función del número de dispositivos Wi-Fi.\\

Para conseguir que el robot realizase giros precisos, era necesario calibrar antes el dispositivo MPU9250 tendiendo en cuenta el error sistemático que se producía al haber materiales ferromagnéticos y campos magnéticos inducidos por los dispositivos electrónicos que nos rodean. A pesar de la calibración, todavía se podían obtener más fluctuaciones como picos aleatorios en las lecturas y para lidiar con estos picos se implementó un filtro de paso bajo sobre los valores ya calibrados.\\

Este trabajo de fin de grado supone un avance respecto a los robots guía convencionales, los cuales son generalmente caros y difíciles de adaptar a diferentes entornos debido a todos los cambios que serían necesarios a nivel de software y esto supondría un gasto de tiempo relevante. Una de las limitaciones del sistema es que al aumentar el número de dispositivos Wi-Fi para localizarse, aumenta el tiempo de respuesta para avanzar de una posición a otra y llegar al objetivo final.\\


Durante la realización del trabajo, se han obtenido conocimientos sobre diferentes habilidades y temas, entre los que se incluyen los siguientes:

\begin{itemize}

 \item \textit{} Adquisición de mayor experiencia y conocimientos sobre nuevas librerías y funciones para la programación en Python.
 \item \textit{} Conocimiento de herramientas de manipulación de imágenes como GIMP.
 \item \textit{} Mayor experiencia y conocimientos sobre el entrenamiento y funcionamiento de redes neuronales.
  \item \textit{} Conocimientos sobre la calibración de sensores como el MPU9250.
   \item \textit{} Desarrollo de nuevas habilidades en Latex.
  \item \textit{} Conocimientos a cerca del funcionamiento y características de piezas \textit{hardware} como el controlador L298N o los motores.
  \item \textit{} Conocimientos sobre la aplicación de modelos de aprendizaje automático para distintas aplicaciones.
  \item \textit{} Conocimientos sobre cómo manejar y controlar la Raspberry a distancia desde otros dispositivos como ordenadores sin cables de por medio.
  
\end{itemize}\


\section{Líneas futuras}
\label{sec:lineas_futuras}

En esta sección se comentan algunas vías para permitir la continuidad de este trabajo y añadir funciones que aún no están disponibles para mejorar el funcionamiento del sistema presente:

\begin{itemize}

 \item \textit{} El modelo usado para entrenar a la red neuronal para clasificar los comandos de voz tiene sobreajuste (overfitting), que ocurre cuando se ajusta demasiado bien a los datos debido a que el dataset es pequeño ya que solamente tiene dos clases. Para solucionarlo, habría que usar más datos de entrenamiento para más clases diferentes y con distintas voces de diferentes personas y edades.
 
 \item \textit{} Usar dispositivos Smartphones en lugar de iPhones, ya que los segundos usan banda de frecuencia 2.4835 GHz que tiene mayor interferencia debido a la saturación de dispositivos que la utilizan esta banda y no se pueden cambiar a la banda de 5GHz ya que esta banda tiene menos congestión y menos interferencias porque menos dispositivos tienden a usarla, lo que mejora la calidad de la señal, mientras que los primeros sí pueden cambiarse a esta banda.
 \item \textit{} Probar este sistema en áreas de mayor tamaño y con más dispositivos Wi-Fi para poder localizarse el robot.
  \item \textit{} Usar dispositivos Bluetooth para comparar la precisión respecto a los dispositivos Wi-Fi.
  \item \textit{} Añadir un altavoz para que el robot se pueda comunicar con el usuario y mantener conversaciones más fluidas sobre a qué lugar el gustaría ir el usuario para que el robot pueda guiarle correctamente.
 \item \textit{} Conseguir que el robot que genere nuevas rutas en el caso de que las haya, cuando detecta un obstáculo como una persona y no se mueve durante mucho tiempo, ya que en este sistema hasta que no se moviera el obstáculo, el robot no avanza.
\end{itemize}\



