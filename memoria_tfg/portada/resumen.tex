\cleardoublepage

\chapter*{Resumen\markboth{Resumen}{Resumen}}

La robótica es un campo extenso que se ha desarrollado con el objetivo de mejorar la calidad de vida en diversos ámbitos. Existen numerosos tipos de robots, cada uno diseñado para cumplir una función específica, en la que suelen igualar o incluso superar el desempeño humano, operando de manera continua sin necesidad de descanso y reduciendo riesgos.

Uno de los retos más importantes de la robótica es el diseño de sistemas de navegación y localización en interiores que sean precisos y eficaces, sobre todo en situaciones donde se emplea el hardware de bajo costo y se dispone de CPUs con capacidades limitadas, ya que los robots convencionales son generalmente más caros y difíciles de adaptar a diferentes entornos debido a todos los cambios que serían necesarios a nivel de software. Los robots de bajo coste son necesarios para dar respuesta a este problema con soluciones que sean fáciles de acceder, y exigiendo algoritmos optimizados que permitan un rendimiento fluido en entornos complejos. También, se necesita que la interfaz de comunicación que utilizan los usuarios sea sencilla e intuitiva, para que se puedan comunicar con el robot de manera natural sin necesidad de conocimientos técnicos avanzados.

Para abordar este problema, se ha diseñado un robot móvil de bajo coste mediante piezas impresas en 3D, siguiendo la filosofía DIY. Mediante diferentes técnicas como los modelos de aprendizaje automático, se ha podido calcular una estimación precisa de los dispositivos de red que son necesarios para que se pueda localizar el robot y a partir de un interfaz HRI amena, se ha sido capaz de controlar al robot por comandos de voz mediante redes neuronales. También se han usado algoritmos de navegación para que el robot pueda calcular trayectorias hacia un objetivo y sensores como el MPU9250 para orientarse mejor y mediante el uso de GIMP se ha diseñado el mapa por el que ha navegado el robot, facilitando así poder llevarlo a otros entornos. Para lograr todo esto en una CPU de baja capacidad de cómputo, se ha usado una Raspberry Pi 4, lo que permite que el sistema sea \textit{low-cost}.

Finalmente, se han realizado numerosos experimentos que evalúan los distintos aspectos técnicos que definen el rendimiento y características del robot final.
