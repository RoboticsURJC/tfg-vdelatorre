% Completa los datos convenientemente en las zonas marcadas con TODO

\documentclass{beamer}
%PARA VISUALIZAR PRESENTACIONN CON NOTAS USAR VISUALIZADOR "pdfpc":
%Para ver las notas, el cronometro y siguente diapo:
% pdfpc --notes=right slides.pdf
% "tecla p": para pausar el cronometro
\mode<presentation> {
  \usetheme{CambridgeUS}
  \usecolortheme{crane} % color naranja
}
\setbeamercolor{titlelike}{parent=structure,bg=yellow!85!orange} % Cambia el color de la caja del título de la página inicial

\setbeamertemplate{navigation symbols}{} % ocultar iconos de navegación
\setbeamerfont{subsection in toc}{size=\small} % reducir tamaño en TOC
\setbeamerfont{date}{size=\tiny}
\usepackage[spanish]{babel}
\usepackage[utf8]{inputenc}
\usepackage{graphicx}
\usepackage{booktabs}
\usepackage{hyperref}
\usepackage{multicol}
\usepackage{pgfpages}
\usepackage{listings}
\usepackage{multimedia}
\usepackage[export]{adjustbox}
\usepackage{outlines} % Para poner bullets tabulados (\1 \2 \3 ...) y no items

\usepackage{array,tabularx} % para tabular leyenda de ecuaciones
\newenvironment{conditions*} % entorno de "leyenda de ecuación"
  {\par\vspace{\abovedisplayskip}\noindent
   \tabularx{\columnwidth}{>{$}l<{$} @{\ : } >{\raggedright\arraybackslash}X}}
  {\endtabularx\par\vspace{\belowdisplayskip}}
  
% USO DE NOTAS
\setbeameroption{hide notes} % Para mostrar u ocultar (hide/show)
%\setbeameroption{show only notes} % Mostrar solo las notas
%\setbeameroption{show notes on second screen=right} % Mostrar notas en otra pantalla
\setbeamertemplate{note page}{ % asi solo muestro el texto de las notas
  \insertnote%
}

%========= TODO: datos internos del documento
\hypersetup{
	pdftitle={Defensa de trabajo de fin de grado de Pon tu nombre},
	pdfauthor={Pon tu nombre},
	pdfsubject={Pon aquí el título completo del trabajo},
	pdfkeywords={teaching, robotics, vision, sensors, actuators, raspberry},
	pdfproducer={pdfLaTeX},
  colorlinks=true,
  linkcolor=blue
}
%=========

%========= TODO: diapositiva de portada
\title[Robot de bajo coste guiado por voz]{Prototipo de robot de bajo coste guiado por voz con técnicas de localización} % El título reducido aparece en la parte inferior de todas las diapositivas
                                         % El título completo aparece solo en la diapositiva de portada
\author[Víctor de la Torre Rosa]{Víctor de la Torre Rosa}
\institute[URJC]
{
\textit{\href{mailto:escribe.tu@correo.es}{\color{blue}{\underline{v.delatorre.2019@alumnos.urjc.es}}}}\\
\vspace{0.5cm}
\includegraphics[width=3cm]{figs/logo-urjc}\\
\vspace{1cm}
Trabajo Fin de Grado
}
\date{xx de xxxxxxx de 20xx}
%=========

%========= COMIENZO DEL DOCUMENTO
\begin{document}

%========= Portada inicial con notas
\begin{frame}[plain] % plain: quita header y footer
\large{\titlepage}
\note[item]{En esta presentación voy a hablar sobre...}
\note[item]{En primer lugar...}
\end{frame}

%========= Licencia
\begin{frame}
\input{licencia.tex}
\end{frame}

%========= Índice o tabla de contenidos (TOC)
\begin{frame}
\frametitle{Contenidos}
%\begin{multicols}{2} % si tengo muchas secciones, lo parte en dos columnas
  \tableofcontents[hideallsubsections] % no muestra subsecciones
%\end{multicols}
\note[item]{La presentaci\'on esta dividida en cuatro partes.}
\end{frame}

%========= Diapositiva "vacía" de comienzo de sección:
\section*{}
\begin{frame}{}
  \centering \Huge
  \emph{Introducción}
\note[item]{Comencemos con la introducción.}
\end{frame}

\section{Introducción}
\subsection{Contexto general}
%========= Diapositiva con imágenes:
\begin{frame}
\frametitle{Robótica móvil}
\centering
\begin{minipage}{0.45\textwidth}
    \centering
    \includegraphics[width=3.4cm]{figs/turtlebot3burguer.jpg}
\end{minipage}
\hfill
\begin{minipage}{0.45\textwidth}
    \centering
    \includegraphics[width=3.4cm]{figs/ttt.png}
\end{minipage}
\end{frame}





%========= Diapositiva con ítems resaltados con colores:
\begin{frame}
\frametitle{Robótica educativa y de bajo coste}
\centering

% Primera fila
\begin{minipage}{0.45\textwidth}
    \centering
    \includegraphics[width=3.4cm]{figs/RoboCup_junior.jpg}
\end{minipage}
\hfill
\begin{minipage}{0.45\textwidth}
    \centering
    \includegraphics[width=3.4cm]{figs/scratch.png}
\end{minipage}

\vspace{0.5cm} % Espacio entre filas

% Segunda fila
\begin{minipage}{0.45\textwidth}
    \centering
    \includegraphics[width=3.4cm]{figs/raspberry4.png} % Cambia por la ruta de la tercera imagen
\end{minipage}
\hfill
\begin{minipage}{0.45\textwidth}
    \centering
    \includegraphics[width=3.4cm]{figs/arduino_uno.jpg} % Cambia por la ruta de la cuarta imagen
\end{minipage}

\end{frame}



\section*{}
\begin{frame}{}
  \centering \Huge
  \emph{Objetivos}
\note[item]{Pasemos ahora a comentar los objetivos que nos hemos con este trabajo.}
\end{frame}

\section{Objetivos}
\begin{frame}
\frametitle{Descripción del problema}
\centering

% Primera fila
\begin{minipage}{0.45\textwidth}
    \centering
    \includegraphics[width=4.4cm]{figs/peper.jpg}
\end{minipage}
\hfill
\begin{minipage}{0.45\textwidth}
    \centering
    \includegraphics[width=3.8cm]{figs/hri.jpg}
\end{minipage}

\vspace{0.5cm} % Espacio entre filas

% Segunda fila
\begin{minipage}{0.45\textwidth}
    \centering
    \includegraphics[width=3.8cm]{figs/robot_rasp.jpg} % Cambia por la ruta de la tercera imagen
\end{minipage}
\hfill
\begin{minipage}{0.45\textwidth}
    \centering
    \includegraphics[width=3.8cm]{figs/rumba.jpg} % Cambia por la ruta de la cuarta imagen
\end{minipage}

\end{frame}

\section{Objetivos}
\begin{frame}
\frametitle{Requisitos}
\begin{enumerate}
\item Coste inferior a 145€.
\item Hardware económico.
\item Disponer de varios puntos de acceso Wi-Fi.
\item Uso de una impresora convencional para las piezas.
\item El sistema debe ser capaz de ejecutar en tiempo real sobre Raspberry.
\item El lenguaje de programación debe ser Python.
\item Batería recargable.
\item Los motores y la batería deben de pesar lo menos posible.
\end{enumerate}
\end{frame}

\begin{frame}
\frametitle{Objetivos específicos}
\begin{enumerate}
\item Explorar diferentes opciones de diseño.
\item Realizar una investigación sobre los distintos componentes tanto hardware como software.
\item Gestionar el control del prototipo desde el ordenador.
\item Poder combinar los datos de navegación y localización sin que intervengan
unos con otros mediante la librería threading.
\item Realizar la calibración necesaria de los sensores.
\item Entrenar una red neuronal con audios para enseñar a la red
a clasificar las órdenes.
\end{enumerate}
\end{frame}

\begin{frame}
\frametitle{Metodología}
\centering
\begin{minipage}{0.45\textwidth}
    \centering
    \includegraphics[width=5.4cm]{figs/git.png}
\end{minipage}
\hfill
\begin{minipage}{0.45\textwidth}
    \centering
    \includegraphics[width=3.2cm]{figs/teams.png}
\end{minipage}
\end{frame}

\section*{}
\begin{frame}{}
  \centering \Huge
  \emph{Plataforma de desarrollo}
\note[item]{Una vez descritos los objetivos, veamos qué hemos hecho para alcanzarlos.}
\end{frame}


\section{Plataforma de desarrollo}
\begin{frame}
\frametitle{Hardware}
\centering

% Primera fila
\begin{minipage}{0.3\textwidth}
    \centering
    \includegraphics[width=2.3cm]{figs/microfono-usb.jpg}
\end{minipage}
\hfill
\begin{minipage}{0.3\textwidth}
    \centering
    \includegraphics[width=3.0cm]{figs/L298N.png}
\end{minipage}
\hfill
\begin{minipage}{0.3\textwidth}
    \centering
    \includegraphics[width=2.6cm]{figs/motor.jpg}
\end{minipage}

\vspace{-0.3cm} % Reduce el espacio entre filas


% Segunda fila
\begin{minipage}{0.3\textwidth}
    \centering
    \includegraphics[width=2.0cm]{figs/bateria.jpg} 
\end{minipage}
\hfill
\begin{minipage}{0.3\textwidth}
    \centering
    \includegraphics[width=1.1cm]{figs/powerbank.jpg} 
\end{minipage}
\hfill
\begin{minipage}{0.3\textwidth}
    \centering
    \includegraphics[width=2.0cm]{figs/rueda_loca.png}
\end{minipage}



% Tercera fila
\begin{minipage}{0.3\textwidth}
    \centering
    \includegraphics[width=2.3cm]{figs/base.png} 
\end{minipage}
\hfill
\begin{minipage}{0.3\textwidth}
    \centering
    \includegraphics[width=2.0cm]{figs/hcsr04.jpg} 
\end{minipage}
\hfill
\begin{minipage}{0.3\textwidth}
    \centering
    \includegraphics[width=1.8cm]{figs/mpu9250.jpg}
\end{minipage}

\vspace{-0.5cm} % Reduce el espacio entre filas

% Imagen centrada en la última fila
\begin{center}
    \includegraphics[width=2.7cm]{figs/raspberry4.png}
\end{center}

\end{frame}


%========= Diapositiva con matemáticas y leyenda (conditions*):
\begin{frame}
\frametitle{Software}

\centering

% Primera fila
\begin{minipage}{0.3\textwidth}
    \centering
    \includegraphics[width=2.0cm]{figs/python.jpeg}
\end{minipage}
\hfill
\begin{minipage}{0.3\textwidth}
    \centering
    \includegraphics[width=3.0cm]{figs/freecad.png}
\end{minipage}
\hfill
\begin{minipage}{0.3\textwidth}
    \centering
    \includegraphics[width=2.3cm]{figs/sklearn.png}
\end{minipage}

\vspace{0.3cm} % Reduce el espacio entre filas


% Segunda fila
\begin{minipage}{0.3\textwidth}
    \centering
    \includegraphics[width=2.3cm]{figs/librosa.png} 
\end{minipage}
\hfill
\begin{minipage}{0.3\textwidth}
    \centering
    \includegraphics[width=2.0cm]{figs/mag.jpg} 
\end{minipage}
\hfill
\begin{minipage}{0.3\textwidth}
    \centering
    \includegraphics[width=2.0cm]{figs/Jupyter.png}
\end{minipage}
% Tercera fila
\begin{minipage}{0.3\textwidth}
    \centering
    \includegraphics[width=2.3cm]{figs/gimp.jpeg} 
\end{minipage}
\end{frame}

\section*{}
\begin{frame}{}
  \centering \Huge
  \emph{Arquitectura hardware}
\note[item]{Una vez descritos los objetivos, veamos qué hemos hecho para alcanzarlos.}
\end{frame}



\section{Arquitectura hardware}
\begin{frame}
\frametitle{Geometría del robot}
\centering
\begin{minipage}{0.45\textwidth}
    \centering
    \includegraphics[width=5.8cm]{figs/base.png}
\end{minipage}
\hfill
\begin{minipage}{0.45\textwidth}
    \centering
    \includegraphics[width=5.5cm]{figs/final_circuito.png}
\end{minipage}

\end{frame}

\section{Arquitectura hardware}
\begin{frame}
\frametitle{Geometría del robot}
\centering
\begin{minipage}{0.45\textwidth}
    \centering
    \includegraphics[width=5.4cm]{figs/rob.jpeg}
\end{minipage}


\end{frame}

\section*{}
\begin{frame}{}
  \centering \Huge
  \emph{Desarrollo software}
\end{frame}

\section{Desarrollo software}
\begin{frame}
\frametitle{Orientación y diseño del mapa}
\centering
\begin{minipage}{0.45\textwidth}
    \centering
    \includegraphics[width=3.6cm]{figs/pepe.jpg}
\end{minipage}
\hfill
\begin{minipage}{0.45\textwidth}
    \centering
    \includegraphics[width=4.5cm]{figs/path1.png}
\end{minipage}
\hfill
\vspace{0.3cm} % Reduce el espacio entre filas
\begin{minipage}{0.45\textwidth}
    \centering
    \includegraphics[width=4.5cm]{figs/bus.png}
\end{minipage}

\end{frame}

\section{Desarrollo software}
\begin{frame}
\frametitle{Interfaz de usuario}
\centering
\begin{minipage}{0.45\textwidth}
    \centering
    \includegraphics[width=5.5cm]{figs/forma_onda.png}
\end{minipage}
\hfill
\begin{minipage}{0.45\textwidth}
    \centering
    \includegraphics[width=5.5cm]{figs/stft_spectogram.png}
\end{minipage}
\hfill
\vspace{0.3cm} % Reduce el espacio entre filas
\begin{minipage}{0.45\textwidth}
    \centering
    \includegraphics[width=5.5cm]{figs/cromagrama.png}
\end{minipage}
\hfill
\begin{minipage}{0.45\textwidth}
    \centering
    \includegraphics[width=5.5cm]{figs/coeficientes_mfc.png}
\end{minipage}
\hfill
\vspace{0.3cm} % Reduce el espacio entre filas
\begin{minipage}{0.45\textwidth}
    \centering
    \includegraphics[width=5.5cm]{figs/mel_spectrogram.png}
\end{minipage}


\end{frame}

\section{Desarrollo software}
\begin{frame}
\frametitle{Localización}
\centering
\begin{minipage}{0.45\textwidth}
    \centering
    \includegraphics[width=5.5cm]{figs/trilateration.png}
\end{minipage}


\end{frame}

\section{Desarrollo software}
\begin{frame}
\frametitle{Localización}
\centering
Una vez se tiene la potencia recibida del AP, para calcular la distancia hay que usar la ecuación:


\begin{equation}
d = 10^{\frac{A - \texttt{RSSI}}{10 \cdot n}}
\label{ec:ec2}
\end{equation}
\texttt{Donde:}
\begin{itemize}
    \item $d$: Distancia estimada en metros.
    \item $A$: Valor RSSI a 1 metro del AP.
    \item \texttt{RSSI}: Potencia recibida.
    \item $n$: Factor de propagación (depende del entorno, generalmente es 2-3 en interiores).
\end{itemize}
\vspace{-0.3cm} % Reduce el espacio entre filas
\begin{equation}
\left\{
	\begin{array}{lcc}
		(x - x_1)^2 + (y - y_1)^2 = (d_1)^2\\
		(x - x_2)^2 + (y - y_2)^2 = (d_2)^2\\
		(x - x_3)^2 + (y - y_3)^2 = (d_3)^2 \\
		(x - x_4)^2 + (y - y_4)^2 = (d_4)^2
	\end{array}
\right.
\label{ec:ec5}
\end{equation}

\end{frame}

\section{Desarrollo software}
\begin{frame}
\frametitle{Navegación (A*)}
\centering


\begin{minipage}{0.45\textwidth}
    \centering
    \includegraphics[width=5.5cm]{figs/astar4.png}
\end{minipage}

\vspace{-0.6cm} % Reduce el espacio entre filas
\begin{equation}
\left\{
	f(n) = g(n) + h(n)
\right.
\label{ec:ec678}
\end{equation}

\texttt{Donde:}
\begin{itemize}
    \item $g(n)$: Coste de cada movimiento desde el nodo inicial hasta el final.
    \item $h(n)$: Es la heurística en la que se estiman los costes futuros. La heurística es una estimación del coste de movimientos futuros.
\end{itemize}

Sumando ambas, se puede estimar el camino de menor coste a recorrer.
\end{frame}

\section*{}
\begin{frame}{}
  \centering \Huge
  \emph{Pruebas y experimentos}
\end{frame}


\section{Pruebas y experimentos}
\begin{frame}
\frametitle{Calibración magnética}
\centering

% Primera fila
\begin{minipage}{0.45\textwidth}
    \centering
    \includegraphics[width=6.5cm]{figs/hard_iron_calibration.png}
\end{minipage}
\hfill
\begin{minipage}{0.45\textwidth}
    \centering
    \includegraphics[width=5.0cm]{figs/soft_iron_calibration.png}
\end{minipage}

\vspace{0.5cm} % Espacio entre filas

% Segunda fila
\begin{minipage}{0.7\textwidth}
    \centering
    \includegraphics[width=8.0cm]{figs/low.png} % Cambia por la ruta de la tercera imagen
\end{minipage}
\hfill


\end{frame}


\section{Pruebas y experimentos}
\begin{frame}
\frametitle{Elección del modelo de aprendizaje automático}
\centering

% Primera fila
\begin{minipage}{0.5\textwidth}
    \centering
    \includegraphics[width=7.5cm]{figs/modelos.png}
\end{minipage}
\hfill

\end{frame}

\section{Pruebas y experimentos}
\begin{frame}
\frametitle{Elección del número de APs}
\centering

% Primera fila
\begin{minipage}{0.45\textwidth}
    \centering
    \includegraphics[width=4.0cm]{figs/dos_apes.png}
\end{minipage}
\hfill
\begin{minipage}{0.45\textwidth}
    \centering
    \includegraphics[width=4.0cm]{figs/tres_apes.png}
\end{minipage}

\vspace{0.3cm} % Espacio entre filas

% Segunda fila
\begin{minipage}{0.45\textwidth}
    \centering
    \includegraphics[width=4.0cm]{figs/cuatro_apes.png} % Cambia por la ruta de la tercera imagen
\end{minipage}
\hfill
\begin{minipage}{0.45\textwidth}
    \centering
    \includegraphics[width=4.0cm]{figs/cuatro_apes_espaciados.png} % Cambia por la ruta de la cuarta imagen
\end{minipage}

\end{frame}


\section{Pruebas y experimentos}
\begin{frame}
\frametitle{Elección del número de APs}
\centering

\begin{table}[H]
\begin{center}
\begin{tabular}{|c|c|c|c|c|}
\hline
\textbf{Nodos} & \textbf{SVC, Linear} & \textbf{SVC,RBF} & \textbf{DTREE} & \textbf{RF} \\
\hline
2 & 28\% & 31\% & 28\% & 24\% \\  
3 & 82\% & 86\% & 78\% & 82\% \\   
4 & 94\% & 94\% & 91\% & 94\% \\   
\hline
\end{tabular}
\label{cuadro:tabla2}
\end{center}
\end{table}

\begin{table}[H]
\begin{center}
\begin{tabular}{|c|c|c|c|c|}
\hline
\textbf{Nodos} & \textbf{SVC, Linear} & \textbf{SVC,RBF} & \textbf{DTREE} & \textbf{RF} \\
\hline
4 & 99\% & 99\% & 96\% & 100\% \\  
\hline
\end{tabular}
\label{cuadro:tabla3}
\end{center}
\end{table}

% Primera fila
\begin{minipage}{0.45\textwidth}
    \centering
    \includegraphics[width=5.2cm]{figs/vals1.png}
\end{minipage}
\hfill
\begin{minipage}{0.45\textwidth}
    \centering
    \includegraphics[width=5.2cm]{figs/vals2.png}
\end{minipage}

\vspace{0.3cm} % Espacio entre filas
\end{frame}


\section*{}
\begin{frame}{}
  \centering \Huge
  \emph{Conclusiones}
\end{frame}


\section{Conclusiones}
\begin{frame}
\begin{block}{Objetivos cumplidos}
\begin{itemize}
\item Diseñar un robot móvil de bajo coste.
\item Usar una CPU de baja capacidad de cómputo.
\item Navegación y localización en interiores.
\item Navegación, orientación y localización en interiores.
\item Es portable a otros entornos interiores.
\item Comunicación con el robot exitosa mediante la red neuronal.
\end{itemize}
\end{block}

\end{frame}

\section{Conclusiones}
\begin{frame}
\begin{block}{Líneas futuras}
\begin{itemize}
\item Usar más datos de entrenamiento para más clases
diferentes y con distintas voces de diferentes personas y edades.
\item Usar dispositivos Smartphones en lugar de iPhones.
\item Probar el sistema en áreas de mayor tamaño y con más dispositivos Wi-Fi.
\item Añadir un altavoz para que el robot se pueda comunicar con el usuario y
mantener conversaciones más fluidas.
\item Conseguir que el robot genere nuevas rutas en el caso de que las haya, cuando
detecta un obstáculo como una persona y no se mueve durante mucho tiempo.
\end{itemize}
\end{block}
\end{frame}

\begin{frame}[plain]
\large{\titlepage}
\note[item]{Y hasta aquí mi exposición.}
\note[item]{Quedo a disposición del tribunal...}
\end{frame}


\end{document}
